\begin{enumerate}
\item A student noted the number of cars passing through a spot on a road for $100$ periods each of $3$ minutes and summarised it in the table given below. Find the mean and median of the following data.\\

	\begin{tabular}{|c|c|c|c|c|c|c|c|c|}
\hline
Number of cars & 0-10 & 10-20 & 20-30 & 30-40 & 40-50 & 50-60 & 60-70 & 70-80\\ 
\hline
Frequency (Periods) & 7 & 14 & 13 & 12 & 20 & 11 & 15 & 8\\ 
\hline

\end{tabular}

\item Computer-based learning $\brak{CBL}$ refers to any teaching methodology that makes use of computers for information transmission. At an elementary school level, computer applications can be used to display multimedia lesson plans. A survey was done on $1000$ elementary and secondary schools of Assam and they were classified by the number of computers they had.

	\begin{figure}[!ht]
		\centering
		\includegraphics[width=\columnwidth]{figs/last1.jpg}
		\caption{}
		\label{fig:enter-label}
	\end{figure}

	\begin{center}
	\begin{tabular}{|c|c|c|c|c|c|}
	\hline
	\textbf{Number of computers} & 1-10 & 11-20 & 21-50 &  51-100 & 101 and more \\
	\hline
	\textbf{Number of Schools} & 250 & 200 & 290 & 180 & 80 \\
	\hline
	\end{tabular}
	\end{center}

	\text One school is chosen at random.Then:
	\begin{enumerate}
		\item  Find the probability that the school chosen at random has more than $100$ computers.
		\item
		\begin{enumerate}
			\item  Find the probability that the school chosen at random has $50$ or fewer computers.
			\item  Find the probability that the school chosen at random has no more than $20$ computers.
		\end{enumerate}
		\item  Find the probability that the school chosen at random has $10$ or less than $10$ computers.
	\end{enumerate}

\item For the following distribution:

\begin{center}
\begin{tabular}{|c|c|c|c|c|c|c|}
\hline
\textbf{Marks Below} & 10 & 20 & 30 & 40 & 50 & 60 \\
\hline
\textbf{Number of Students} & 3 & 12 & 27 & 57 & 75 & 80 \\
\hline
\end{tabular}
\end{center}

The modal class is:

\begin{enumerate}
    \item $10-20$
    \item $20-30$
    \item $30-40$
    \item $50-60$
\end{enumerate}
\item  Two dice are thrown together. The probability of getting the difference of numbers on their upper faces equal to 3 is:

	\begin{enumerate}
	\item  $\frac{1}{9}$
	\item $\frac{2}{9}$
	\item  $\frac{1}{6}$
	\item  $\frac{1}{12}$
\end{enumerate}
\item A Card is drawn at random from a well-shuffled pack of 52 cards.The probability that the card drawn is not an ace is:
\begin{enumerate}
	\item $\frac{1}{13}$
	\item  $\frac{9}{13}$
	\item $\frac{4}{13}$
	\item $\frac{12}{13}$
\end{enumerate}
\item{DIRECTIONS:} In questions number 19 and 20, a statement of Assertion (A) is followed by a statement of Reason (R). Choose the correct option out of the following:
Assertion (A): The probability that a leap year has 53 Sundays is $\frac{2}{7}$.

Reason (R): The probability that a non-leap year has 53 Sundays is $\frac{5}{7}$.

\begin{enumerate}
    \item Both Assertion (A) and Reason (R) are true and Reason (R) is the correct explanation of Assertion (A).
    \item Both Assertion (A) and Reason (R) are true and Reason (R) is not the correct explanation of Assertion (A).
    \item Assertion (A) is true but Reason (R) is false.
    \item Assertion (A) is false but Reason (R) is true.
\end{enumerate}
\end{enumerate}
